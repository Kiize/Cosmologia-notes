\documentclass[10pt,a4paper]{article}
\usepackage[utf8]{inputenc}
\usepackage[T1]{fontenc}
\usepackage{amsmath}
\usepackage{amsfonts}
\usepackage{amssymb}
\usepackage{graphicx}
\usepackage{amsthm}		%teoremi
\usepackage{tikz}		%disegnare immagini
\usepackage{marginnote}

\let\oldmarginpar\marginpar
\renewcommand\marginpar[1]{\-\oldmarginpar[\raggedleft\footnotesize #1]%
	{\raggedright\footnotesize #1}}

%THEOREMS, ...
\newtheoremstyle{break}%
{}{}%
{}{}%
{\itshape}{}%
{\newline}{}


\theoremstyle{break}
\newtheorem{ex}{Esempio}
\newtheorem{defn}{Definizione}

\theoremstyle{remark}
\newtheorem{oss}{Osservazione}
\newtheorem{idea}{Idea}

\theoremstyle{definition}
\newtheorem{lem}{Lemma}
\newtheorem{teo}{Teorema}
\newtheorem{prop}{Proprietà}
\newtheorem{coroll}{Corollario}

%NEW COMMANDS
\newcommand{\bbx}{\mathbb{X}}
\newcommand{\bbr}{\mathbb{R}}
\newcommand{\bbn}{\mathbb{N}}
\newcommand{\ra}{\Rightarrow}
\newcommand{\Lagr}{\mathcal{L}}

\title{Cosmologia del primo universo}	


\usepackage[top=2cm, bottom=1.3cm, left=1cm, right=5cm, heightrounded,
marginparwidth=4.6cm, marginparsep=3mm]{geometry} %to show the margin

\begin{document}
\maketitle

\section{Richiami di RG}
Nello spazio-tempo di Minkowski possiamo definire un prodotto scalare 
\[
A \cdot B = A^\mu B^\mu = g_{\mu \nu} A^\mu B^\nu
\]
ed un elemento di lunghezza invariante

\marginpar{Utilizziamo come convenzione $\eta_{\mu \nu} = diag\{-, +, +, +\}$}

\[
ds^2 = g_{\mu \nu}dx^\mu dx^\nu
\]


\subsection{Principio di equivalenza}
Versione debole: è sempre possibile ridursi localmente ad uno spazio piatto in cui le leggi della meccanica siano uguali a quelle in assenza di gravità. 

\begin{ex}
	Consideriamo una particella in un ascensore in caduta libera. La particella si comporta come un sistema inerziale 
	\[
	F_{tot} = m_g g - m_i a
	\]
	\[
	g = a \ra m_i = m_a
	\]
	poichè sulla particella non agiscono forze esterne. Allora la massa inerziale è uguale a quella gravitazionale.
\end{ex}

Versione forte: L'equivalenza si estende a tutte le leggi fisiche, non solo quelle della meccanica.

\begin{ex}[Esperimento ideale della propagazione della luce]
	Consideriamo un ascensore che si muove in caduta libera ed una particella che si muove trasversalmente al suo interno. Per il principio di equivalenza il moto della particella deve essere rettilineo uniforme.
	\\
	Nel sistema del laboratorio, affinchè la particella attraversi la metà dell'ascensore, deve avere una traiettoria curva. Possiamo perciò dedurre che cambiando le leggi del moto cambiano le traiettorie.
	\\
	Nel sistema in caduta libera, usando le coordinate $\{\xi^\alpha\}$, si ha che
	\[
	\frac{d^2\xi^\alpha}{d\tau^2} = 0
	\]
	dove $\tau$ è il tempo nel sistema solidale. Se cambiamo coordinate $\xi^\alpha \to x^\mu$ allora l'equazione del moto diventa
	\[
	0 = \frac{d}{d\tau}(\frac{\partial\xi^\alpha}{\partial x^\mu} \frac{\partial x^\mu}{\partial \tau}) = \frac{\partial^2 \xi^\alpha}{\partial x^\mu \partial x^\nu} \frac{\partial x^\mu}{\partial \tau} \frac{\partial x^\nu}{\partial \tau} + \frac{\partial\xi^\alpha}{\partial x^\mu}\frac{\partial^2 x^\mu}{\partial \tau^2}	
	\]
	Moltiplicando per $\frac{\partial x^\lambda}{\partial \xi^\alpha}$ otteniamo che 
	\[
	\frac{\partial^2 x^\mu}{\partial \tau^2} \underbrace{ \frac{\partial\xi^\alpha}{\partial x^\mu}\frac{\partial x^\lambda}{\partial \xi^\alpha} }_{ \delta_\mu^\lambda } + \underbrace{\big( \frac{\partial^2 \xi^\alpha}{\partial x^\mu \partial x^\nu} \frac{\partial x^\lambda}{\partial \xi^\alpha} \big)}_{\text{Simboli di Christoffel}, \Gamma^\lambda_{\mu \nu} }  \frac{\partial x^\mu}{\partial \tau} \frac{\partial x^\nu}{\partial \tau} = 0
	\]
	Otteniamo così l'equazione della geodetica
	
	\[
	\boxed{\frac{d^2 x^\lambda}{d \tau^2} +  \Gamma^\lambda_{\mu \nu}\frac{\partial x^\mu}{\partial \tau} \frac{\partial x^\nu}{\partial \tau} = 0 }
	\]
\end{ex}

\begin{oss}
	Nel caso dei fotoni non possiamo lavorare con $\tau$, tuttavia si può dimostrare che giungiamo alla stessa equazione.
\end{oss}

Nel sistema in caduta libera poi
\[
ds'^2 = \eta_{\alpha \beta}d\xi^\alpha d\xi^\beta = \eta_{\alpha \beta} \frac{\partial\xi^\alpha}{\partial x^\mu} dx^\mu \frac{\partial\xi^\beta}{\partial x^\nu} dx^\nu = \big( \eta_{\alpha \beta} \frac{\partial\xi^\alpha}{\partial x^\mu}\frac{\partial\xi^\beta}{\partial x^\nu} \big) dx^\mu dx^\nu = g_{\mu \nu} dx^\mu dx^\nu = ds^2 
\]
Allora dall'invarianza di $ds^2$ segue che 
\[
\eta_{\alpha \beta} \frac{\partial\xi^\alpha}{\partial x^\mu}\frac{\partial\xi^\beta}{\partial x^\nu} = g_{\mu \nu}
\]

\begin{ex}[Redshift gravitazionale]
	Consideriamo un sistema di due punti A e B in caduta libera attraversati da un segnale di frequenza $\nu$. Per il principio di equivalenza debole sappiamo che, nel sistema solidale, in B misuriamo la stessa frequenza che misuriamo in A $\nu_B = \nu_A$, tuttavia nel sistema del laboratorio B si allontana, abbiamo dunque un effetto Doppler che il campo gravitazionale deve compensare.
	\\
	Nel sistema solidale $d\xi^i = 0$, allora
	\[
	ds = c d\tau = \sqrt{\eta_{\alpha \beta} d\xi^\alpha d\xi^\beta } = \sqrt{ \eta_{\alpha \beta} \frac{\partial\xi^\alpha}{\partial x^\mu}\frac{\partial\xi^\beta}{\partial x^\nu} dx^\mu dx^\nu  } = \sqrt{ g_{\mu \nu} dx^\mu dx^\nu }
	\]
	Dividendo per $cdt$ otteniamo
	\[
	\frac{d\tau}{dt} = \frac{1}{c} \sqrt{ g_{\mu \nu} \frac{dx^\mu}{dt} \frac{dx^\nu}{dt} }
	\]
	Se consideriamo l'orologio a riposo 
	\marginpar{$g_{00} < 1 \ra \Delta t > \Delta \tau$: dilatazione dei tempi, redshift gravitazionale}
	\[
	\frac{d\tau}{dt} = \sqrt{g_{00}}
	\]
	Allora
	\[
	\frac{\nu_2}{\nu_1} = \frac{dt_1}{dt_2} = \sqrt{\frac{g_{00}(x_2)}{g_{00}(x_1)}}
	\]
	dove $\nu_i \propto \frac{1}{dt_i}$ è la frequenza misurata nel laboratorio nel punto i-esimo.
\end{ex}

Consideriamo ora il limite newtoniano (piccole velocità) allora 
\[
g_{\mu \nu} = \eta_{\mu \nu} + h_{\mu \nu}
\]
detto limite di campo debole,con $g_{0i} = 0$ per invarianza temporale e $|h_{\mu \nu}| \ll 1$.
\\
In questo limite l'equazione geodetica diventa 
\[
\frac{d^2 x^i}{dt^2} = - \frac{c^2}{2} \vec{\nabla}h_{00}
\]
che deve ricondursi all'equazione di Poisson per il campo gravitazionale, cioè
\[
\frac{d^2 x^i}{dt^2} = - \vec{\nabla}\Phi
\]
con $\Phi$ potenziale gravitazionale. Allora $h_{00} = \frac{2\Phi}{c^2}$ dove le BC determinano la costante che risulta quindi essere 0. Inoltre osserviamo che $\frac{\Delta\nu}{\nu} = \Delta\Phi$.

Richiamiamo adesso alcune proprietà. Detta $g = det{g_{\mu \nu}}$, allora la forma volume invariante è 
	\[
	\sqrt{-g}d^4x = \sqrt{-g'} d^4 x'
	\]
e 
\[
\partial_\alpha g = g g^{\mu \nu} \partial_\alpha g_{\mu \nu}
\]
Inoltre i simboli di Christoffel sono 
\[
\Gamma^\mu_{\rho \nu} = \frac12 g^{\mu \lambda} (\partial_\rho g_{\nu \lambda} + \partial_\nu g_{\rho \lambda} - \partial_\lambda g_{\nu \rho}  )
\]
ed hanno la proprietà che
\[
\Gamma^\mu_{\mu \nu = \partial_\nu}(\ln\sqrt{-g})
\]
Infine ricordiamo che la derivata covariante della metrica è nulla, infatti se $A_\mu = g_{\mu \nu} A^\nu$ allora
\[
D_\lambda A_\mu = D_\lambda( g_{\mu \nu} A^\nu) = (D_\lambda g_{\mu \nu}) A^\nu + g_{\mu \nu} D_\lambda A^\nu = (D_\lambda g_{\mu \nu}) A^\nu + g_{\mu \nu}(D A)^\nu_\lambda = (D_\lambda g_{\mu \nu}) A^\nu + D_\lambda A_\mu
\]
quindi $D_\lambda g_{\mu \nu} = 0 \quad \forall \mu, \nu, \lambda$.

\subsection{Isometrie}
Le isometrie sono quelle trasformazioni che lasciano invariata la metrica. In generale una metrica trasforma nel modo seguente
\[
g_{\mu \nu}(x) = \frac{\partial x'^\rho}{\partial x^\mu} \frac{\partial x'^\sigma}{\partial x^\nu} g'_{\rho \sigma}(x')
\]
allora un isometria è una trasformazione tale per cui
\[
g'_{\rho \sigma}(x) = g_{\rho \sigma}(x)
\]
Consideriamo una trasformazione di coordinate infinitesima 
\[
x'^\mu = x^\mu + \varepsilon \xi^\mu
\]
allora se la trasformazione è un'isometria si ha che, espandendo in serie di Taylor
\[
g'_{\rho \sigma}(x') = g_{\rho \sigma}(x') = g_{\rho \sigma}(x) + (\partial_\tau g_{\rho \sigma}(x) )\varepsilon \xi^\tau
\]
Sostituendo questo risultato nella trasformazione della metrica, al prim'ordine abbiamo che

\begin{align*}
	g_{\mu \nu} &= (\delta^\rho_\mu + \varepsilon\partial_\mu\xi^\rho) (\delta^\sigma_\nu + \varepsilon\partial_\nu\xi^\sigma)(g_{\rho \sigma} + (\partial_\tau g_{\rho \sigma} )\varepsilon \xi^\tau) \\
	&= (\delta^\rho_\mu \delta^\sigma_\nu + \varepsilon(\delta^\sigma_\nu\partial_\mu\xi^\rho + \delta^\rho_\mu \partial_\nu \xi^\sigma ))(g_{\rho \sigma} + (\partial_\tau g_{\rho \sigma})\varepsilon\xi^\tau) \\
	&= g_{\mu \nu} + \varepsilon((\partial_\tau g_{\mu \nu})\xi^\tau + g_{\nu \rho}\partial_\mu\xi^\rho + g_{\mu \sigma}\partial_\nu\xi^\sigma ) \\
\end{align*}

Allora
\[
\xi^\tau\partial_\tau g_{\mu \nu}  + g_{\nu \rho}\partial_\mu\xi^\rho + g_{\mu \sigma}\partial_\nu\xi^\sigma = 0
\]

Usando adesso Leibniz si ha che
\[
\partial_\mu(g_{\rho \nu}\xi^\rho) = \partial_\mu \xi_\nu = ((\partial_\mu g_{\rho \nu})\xi^\rho + g_{\rho \nu}\partial_\mu\xi^\rho)
\]
Quindi
\[
\xi^\tau\partial_\tau g_{\mu \nu} + \partial_\mu \xi_\nu - (\partial_\mu g_{\rho \nu})\xi^\rho +  + \partial_\nu \xi_\mu - (\partial_\nu g_{\sigma \mu})\xi^\sigma = 0 
\]
che possiamo riscrivere come, cambiando nome agli indici muti
\[
\xi^\tau(\partial_\tau g_{\mu \nu} - \partial_\mu g_{\tau \nu} - \partial_\nu g_{\tau \mu}) + \partial_\mu \xi_\nu + \partial_\nu \xi_\mu = 0
\]
Abbassiamo gli indici con la metrica $\xi^\tau = g^{\lambda \tau} \xi_\lambda$ ed otteniamo
\[
\partial_\mu \xi_\nu + \partial_\nu \xi_\mu - 2 \xi_\lambda \Gamma^\lambda_{\mu \nu} = 0
\]
o analogamente
\[
\boxed{
D_\mu \xi_\nu + D_\nu \xi_\mu = 0
}
\]
detta Equazione di Killing.
\begin{defn}[Vettori di Killing]
	I vettori che soddisfano l'equazione di Killing sono detti vettori di Killing e quindi individuano delle isometrie.
\end{defn}

\begin{defn}[Spazio massimamente simmetrico]
	Uno spazio N dimensionale si dice massimamente simmetrico se il numero dei vettori di Killing è $\frac{N(N+1)}{2}$
\end{defn}

\begin{oss}
	Gli spazi massimamente simmetrici sono completamente definiti da una metrica e da una curvatura k costante nello spazio.
\end{oss}

\marginpar{Q. ?}

Possiamo scrivere i simboli di Christoffel come 
\[
\Gamma^\mu_{\nu \lambda} = k x^\mu g_{\lambda \nu}
\]
allora in assenza di campi gravitazionali 
\[
\frac{d^2 x^\mu}{d\tau^2} - k x^\mu = 0
\]

\begin{defn}[Tensore di Riemann]
	\[
	R^\mu_{~\nu \alpha \beta} = \partial_\alpha \Gamma^\mu_{\nu \beta} \partial_\beta \Gamma^\mu_{\nu \alpha} + \Gamma^\mu_{\alpha \lambda}\Gamma^\lambda_{\beta \nu} - \Gamma^\mu_{\beta \lambda}\Gamma^\lambda_{\alpha \nu}
	\]
\end{defn}
e rispetta l'identità di Bianchi
\[
R^\mu_{~\nu \alpha \beta} + R^\mu_{~ \beta \nu \alpha } + R^\mu_{~\alpha \beta \nu  } = 0
\]
\begin{defn}[Tensore di Ricci]
	\[
	R_{\nu \beta} = R^\mu_{~\nu \mu \beta}
	\]
\end{defn}

\begin{defn}[Curvatura scalare]
	\[
	R = R^\mu_\mu
	\]
\end{defn}
\begin{defn}
	\[
	G_{\mu \nu} = R_{\mu \nu} - \frac12 g_{\mu \nu} R
	\]
\end{defn}
Ricordiamo una utile proprietà
\[
D^\mu R_{\mu \nu} - \frac12 D_\nu R = 0
\]
da cui deriva che 
\[
D^\mu G_{\mu \nu} = 0
\]
\subsection{Dinamica}
Studiamo adesso l'azione del campo gravitazionale $S_g$. Questa deve essere uno scalare e l'unico scalare fin'ora definito è R, allora possiamo costruire 
\[
\boxed{
S_g \propto \int R \sqrt{-g}d^4x
}
\]
con $ \sqrt{-g}d^4x$ intervallo invariante.
\begin{oss}
	La definizione di $S_g$ è a meno di una costante, infatti possiamo anche definire $S_g \propto \int (R - 2\Lambda) \sqrt{-g}d^4x$ da cui segue naturalmente il termine cosmologico.
\end{oss}

Per i campi di materia abbiamo invece un'azione del tipo
\[
S_m = \frac1c \int \Lagr_m \sqrt{-g}d^4x
\]
In assenza di campo gravitazionale
\[
\delta S_m = 0 \ra D_\alpha T_{\mu \nu} = 0
\]
cioè il tensore di energia-impulso si conserva.
\\
In presenza, invece, sia di un campo di materia che di un campo gravitazionale abbiamo che 
\[
\delta(S_m + S_g) = 0 \not\ra D^\mu T_{\mu \nu} = 0 
\]
Richiedendo poi $\delta(S_m + S_g) = 0$ otteniamo le equazioni di Einstein
\[
\boxed{
R_{\mu \nu} - \frac12 g_{\mu \nu}R = 8\pi G T_{\mu \nu}
}
\]
Usando il fatto che $R = -8 \pi G T,\, T = T^\mu_\mu$ possiamo riscriverle come
\[
R_{\mu \nu} = 8\pi G(T_{\mu \nu} - \frac12 g_{\mu \nu}T)
\]

\section{Cosmologia}
Il principio cosmologico asserisce che l'universo è isotropo e omogeneo su larghe scale.
\\
In un sistema di questo tipo possiamo introdurre una metrica 
\[
ds^2 = -dt^2 + q^2(t) \tilde{g}_{ij} dx^i dx^j
\]
dove il tempo è fattorizzato nel fattore di scala $q(t)$ per isotropia e quindi $\tilde{g}$ è indipendente dal tempo.

\section{Paradosso di Olbers o del cielo stellato}
Per omogeneità ed isotropia, in ogni punto del cielo ci dovrebbe essere una stella, ma allora il flusso di stelle
\[
\Phi_S = \int_{0}^{\infty} \frac{L_S}{4 \pi r^2} n_S 4 \pi r^2 dr
\]
dove: \\
$\Phi_S$ è il flusso di stelle \\
$L_S $ è la luminosità della stella \\
$n_S$ è la densità di stelle che assumiamo uniforme.\\
Allora 
\[
\Phi_S = L_S n_S \int_{0}^{\infty} dr \to \infty
\]
Ci dovrebbe arrivare dunque una quantità di luce infinita.
\\
Olbers ipotizzò l'esistenza di polvere nel mezzo intergalattico (universo) che assorbisse la radiazione. Tuttavia la polvere si comporta come corpo nero quindi si porta in equilibrio termico con le stelle e riemette la stessa radiazione.
\\
Per risolvere il problema ricorriamo all'espansione dell'universo. Le stelle infatti si allontanano, quindi la luce che emettono subisce il redshift ed arriva a noi come luce nell'infrarosso, non nel visibile.

\section{Metrica FRW}
Consideriamo una sfera 
\[
x_1^2 + x_2^2 + x_3^2 = a^2
\]
immersa in uno spazio euclideo con metrica
\[
dl^2 = dx_1^2 + dx_2^2 + dx_3^2
\]
Abbiamo allora che 
\[
dx_3 = -\frac{x_1 dx_1 + x_2 dx_2}{\sqrt{a^2 - x_1^2 - x_2^2}}
\]
Sostituendo nella metrica otteniamo
\[
dl^2 = dx_1^2 + dx_2^2 + \frac{(x_1 dx_1 + x_2 dx_2)^2}{a^2 - x_1^2 - x_2^2} = d\vec{x}^2 + \frac{(\vec{x} d\vec{x})^2}{a^2 - \vec{x}^2}
\]
con $\vec{x} = (x_1, x_2)$ \\
Analogamente immergendo uno spazio iperbolico 
\[
x_1^2 + x_2^2 - x_3^2 = a^2
\]
in uno spazio minkowskiano
\[
dl^2 = dx_1^2 + dx_2^2 - dx_3^2
\]
otteniamo
\[
dl^2 =  d\vec{x}^2 - \frac{(\vec{x} d\vec{x})^2}{a^2 + \vec{x}^2}
\]
Infine possiamo vedere uno spazio piatto come una sfera in cui $a \to \infty$.
\\
Ridefinendo allora $x' = x/a$ otteniamo
\[
dl^2 =  a^2( d\vec{x'}^2 + k \frac{(\vec{x'} d\vec{x'})^2}{1 - k \vec{x}'^2} )
\]
con 
\[
k = \begin{cases}
	1,& \; \text{sfera} \\
	0,& \; \text{piano} \\
	-1,& \; \text{iperbole} \\
\end{cases}
\]
Lavorando invece in coordinate cilindriche $(r', \theta, x_3)$ otteniamo per la sfera
\[
dl^2 = \frac{a^2 dr'^2}{a^2 - r'^2} + r'^2 d\theta^2
\] 
oppure riscalando il raggio con $r = r'/a$
\[
dl^2 = a^2(\frac{dr^2}{1 - r^2} + r^2 d\theta^2)
\]
Per una superficie iperbolica possiamo mandare $a \to ia$, mentre per una superficie piatta $a \to \infty$. Quindi in forma compatta, usando il raggio riscalato
\[
dl^2 = a^2(\frac{dr^2}{1 - k r^2} + r^2 d\theta^2)
\]
Analogamente in coordinate sferiche $(\chi, \theta , \varphi), \; \chi \in [0, \pi]$
otteniamo
\[
dl^2 = a^2 [d \chi^2 + f_k^2 d \Omega^2]
\]
con 
\marginpar{Per isotropia tutte le dipendenze dal tempo devono essere uguali altrimenti l'universo si deformerebbe}
\[
f_k = \begin{cases}
	\sin \chi, & k=1 \\
	 \chi, & k=0 \\
	\sinh \chi, & k=-1 \\
\end{cases}
\]

Otteniamo così la metrica di Friedmann-Robertson-Walker (FRW)
\[
\boxed{
	ds^2 = -c^2 dt^2 + dl^2 = -c^2 dt^2 + a^2(t)\bigg\{\frac{dr^2}{1 - kr^2} + r^2 d\Omega^2\bigg\}
}
\]
Riscalando il tempo con $d\eta = \frac{c}{a(t)} dt$ otteniamo 
\[
ds^2 = a^2(\eta) \bigg\{d\eta^2 + \frac{dr^2}{1 - kr^2} + r^2 d\Omega^2 \bigg\}
\]
dove $\eta$ è detto tempo conforme.
\\
Calcoliamo adesso i simboli di Christoffel per la metrica FRW ricordando la definizione data in precedenza. Gli unici non nulli sono
\marginpar{Abbiamo usato che $\partial_i g_{\mu \nu} = 0 \; \forall \mu, \nu$, $g^{00} = -1$, $g_{ij} = a^2 \gamma_{ij}$ dove $\gamma$ è la parte spaziale della metrica indipendente dal tempo.}
\[
\Gamma^0_{ij} = \frac12 g^{00} \bigg[\partial_i g_{0j} + \partial_j g_{0i} - \partial_0 g_{ij}\bigg] = -\frac12 g^{00} \partial_0 g_{ij} = a \dot{a} \gamma_{ij} = \frac{\dot{a}}{a}g_{ij}
\]
\marginpar{Abbiamo usato che $g^{il} = \frac{1}{a^2}\gamma^{il}$, che si ricava usando la relazione $g^{il}g_{lj} = \delta^i_j$}
\[
\Gamma^i_{0j} = \frac12 g^{il} \bigg[\partial_j g_{l0} + \partial_0 g_{lj} - \partial_l g_{0j} \bigg] = \frac12 g^{il} \partial_0 g_{lj} = \frac{\dot{a}}{a} \delta^i_j
\]
\begin{ex}
	In questa metrica, una particella inizialmente a riposo rimane ferma, infatti scrivendo l'equazione geodetica
	\[
	\frac{d^2 x^i}{d\tau^2} = -\Gamma^i_{00} \frac{dx^0}{d\tau} \frac{dx^0}{d\tau} = 0
	\]
	poichè sia $\Gamma^i_{00}$ che $\frac{dx^i}{d\tau}$ sono nulli.
\end{ex}

\begin{ex}
	Calcoliamo la distanza propria tra due punti nella metrica FRW. Per arbitrarietà del sistema di riferimento sotto rotazioni e traslazioni, possiamo ricondurre il problema al caso unidimensionale e quindi calcolare la distanza tra un punto nell'origine ed uno a distanza $r_1$. La distanza propria $d(r, t)$ è allora
	\[
	d(r, t) = \int_0^{r_1} dl(r) = \int_0^{r_1} \frac{a(t)}{\sqrt{1 - k r^2}} dr = a(t) \begin{cases}
		\arcsin(r_1),& k=1 \\
		r_1,& k=0 \\
		\text{arcsinh}(r_1),& k=-1 \\
	\end{cases}
	\]
	Quindi i punti sono fermi, ma la loro distanza cambia nel tempo.
\end{ex}

Introduciamo adesso il redshift cosmologico.
\\
Consideriamo una particella in moto e prendiamo il sistema di riferimento in modo tale che la particella sia in prossimità dell'origine, cioè vogliamo che $\gamma_{ij} = \delta_{ij} + o(x^2)$. Questo implica che $\Gamma^i_{jl} \propto g^i_{jl} \simeq 0$. Allora
\marginpar{Il 2 è dovuto al fatto che dobbiamo considerare anche $\Gamma^i{j0}$}
\[
\frac{d^2 x^i}{d\tau^2} = -2 \Gamma^i_{0j} \frac{dx^j}{d\tau} \frac{dx^0}{d\tau} = -2 \frac{\dot{a}}{a} \frac{dx^i}{d\tau} \frac{dx^0}{d\tau}
\]
Se moltiplichiamo a destra e a sinistra per $\frac{d\tau}{dt}$ otteniamo
\marginpar{$\frac{dx^0}{dt} = c = 1$ in unità naturali}
\[
\frac{d}{dt}\bigg(\frac{dx^i}{d\tau}\bigg) = -2 \frac{\dot{a}}{a} \frac{dx^i}{d\tau}\frac{dx^0}{dt} = -\frac{2}{a} \frac{da}{dt} \frac{dx^i}{d\tau}
\]
quindi 
\[
\frac{dx^i}{d\tau} \propto \frac{1}{a^2}
\]
La quantità di moto è poi
\[
p^2 = p_i p^i = m^2 u_i u^i = m^2 g_{ij} \frac{dx^i}{d\tau} \frac{dx^j}{d\tau} \propto a^2 \frac{1}{a^2} \frac{1}{a^2} \propto \frac{1}{a^2}
\]
Tuttavia  $p$ è anche $p = \frac{\hbar}{\lambda}$ allora $\lambda \propto a$.
\begin{defn}[Redshift]
	Definiamo il redshift z come $z = \frac{\lambda_0 - \lambda}{\lambda} = \frac{\lambda_0}{\lambda}$
\end{defn}

Si ha allora che 
\[
\frac{\lambda}{\lambda_0} = \frac{1}{1 + z} = \frac{a(t)}{a(t_0)}
\]
Per parametrizzare il tempo in cosmologia possiamo utilizzare allora il redshift.
\marginpar{$H_0 = 100 h \frac{km}{s\cdot Mpc} \sim 75 \frac{km}{s\cdot Mpc}$}
\begin{defn}[Raggio di Hubble]
	$R_H = \frac{c}{H_0}$ è la distanza massima oltre la quale non può arrivare nessun segnale.
\end{defn}

\section{Legge di Hubble- Le Maitre}
Espandiamo $a(t)$ in serie di Taylor attorno al tempo attuale $t_0$
\marginpar{t: tempo di emissione \\ $t_0$: tempo attuale o di osservazione}
\[
a(t) = a(t_0) \bigg[1 + \frac{\dot{a}}{a} (t - t_0) \bigg]
\]
Lo sviluppo è valido se $t -  t_0 \ll H_0^{-1}$. Se scriviamo $a(t)$ usando il redshift otteniamo
\[
1 + z = \frac{a(t_0)}{a(t)} = \frac{1}{[1 + H_0 (t - t_0) } = 1 + H_0(t - t_0)
\]
\[
\boxed{
\ra z =  H_0(t - t_0) = H_0 d
}
\]
che è la Legge di Hubble-Le Maitre.\\
Ci chiediamo cosa accada alla densità di particelle n nella metrica FRW.\\
Per isotropia $<v^i> = $(media spaziale) altrimenti ci sarebbe una corrente di particelle in una certa direzione. Allora una qualsiasi corrente $J^\mu = n u^\mu$ avrà $J^i = 0$
\marginpar{$u^\mu$: 4-velocità mediata }
In generale la 4-corrente è conservata, cioè
\[
D_\mu J^\mu = 0 = \partial_\mu J^\mu + \Gamma^\mu_{\mu \nu}  J^\nu = \partial_0 J^0 + \Gamma^i_{i0}J^0
\]
\[
\ra \partial_0 J^0 = - 3\frac{\dot{a}}{a} J^0 \ra \partial_0 n = -3 \frac{\dot{a}}{a}n \ra \frac{dn}{n} = - 3 \frac{da}{a}
\]
Quindi
\[
n(t) = n(0) \bigg(\frac{a(0)}{a(t)}\bigg)^3
\]
cioè se $a(t)$ aumenta, $n(t)$ diminuisce.

\section{Tensore energia-impulso}
Al fine di rispettare il principio cosmologico, ogni fluido deve essere perfetto, cioè deve apparire, nel sistema comovente, omogeneo e isotropo. Questa condizione si traduce nella scelta di un tensore energia-impulso nel sistema comovente del tipo

\[
T_{\mu \nu} = \text{diag}\{\rho, p, p, p\}
\]
\marginpar{$\rho$: densità di energia \\ p: pressione}
che possiamo scrivere nella forma
\[
\boxed{
T_{\mu \nu} = (\rho + p)u_\mu u_\nu + p g_{\mu \nu}
}
\]
la quale è esplicitamente covariante.
\begin{prop}
	\[
	T^\mu_\mu = 3p - \rho \equiv T
	\]
	\[
	D^\mu T_{\mu \nu} = 0
	\]
\end{prop}

Dalla conservazione di $T_{\mu \nu}$ nella metrica FRW otteniamo l'equazione di conservazione dell'energia.
\begin{proof}
	\begin{align*}
	D_\mu T^{\mu 0} = 0 &= \partial_\mu T^{\mu 0} + \Gamma^0_{\mu \lambda} T^{\mu \lambda} + \Gamma^\mu_{0 \lambda} T^{\lambda 0} \\
	&= \partial_0 T^{00} + \Gamma^0_{ij} T^{ij} + \Gamma^i_{i0} T^{00} \\
	&= \partial_t \rho + 3\frac{\dot{a}}{a}(\rho + p) = 0 \\
	\end{align*}
\end{proof}
L'equazione di conservazione è dunque
\[
	\boxed{
	\frac{\partial \rho}{\partial t} + 3\frac{\dot{a}}{a}(\rho + p) = 0 
	}
\]
L'espressione della conservazione del tensore energia-impulso è covariante, quindi le leggi di conservazione valgono in tutti i sistemi di riferimento.

\section{Equazioni di Friedmann}
Se partiamo dalle equazioni di Einstein
\[
R_{\mu \nu} = 8 \pi G (T_{\mu \nu} - \frac12 g_{\mu \nu} T)
\]
e calcoliamo il tensore di Ricci per la metrica FRW otteniamo le equazioni di Friedmann
\[
\boxed{
	\frac{\ddot{a}}{a} = - \frac{4 \pi G}{3} (\rho + 3p)
}
\]
\marginpar{La seconda equazione è ottenuta sviluppando in un intorno di $x \simeq 0$ tuttavia, per covarianza, le equazioni sono valide globalmente}
\[
\boxed{
	k + \dot{a}^2 = \frac{8 \pi G}{3} a^2 \rho
}
\]

\begin{oss}
	Derivando rispetto al tempo la seconda equazione di Friedmann ritroviamo la conservazione del tensore energia-impulso
\end{oss}
Se supponiamo adesso un'equazione di stato per il fluido del tipo $p = w \rho$, sostituendo nell'equazione per la conservazione dell'energia otteniamo
\[
\frac{d \rho}{\rho} = -3(1 + w)\frac{da}{a} \\
\ra \rho \propto a^{-3(1 + w)}
\]
Possiamo così distinguere 3 casi importanti
\[
w = 0: \qquad \rho \propto a^{-3} \qquad \text{Materia non relativistica}
\]
\[
w = \frac13 : \qquad \rho \propto a^{-4} \qquad \text{Radiazione}
\]
\[
w = -1: \qquad \rho = c.te \qquad \text{Costante cosmologica}
\]
Integrando, poi, la seconda equazione di Friedmann a $k = 0$ possiamo ottenere la dipendenza temporale di a e quindi di $\rho$.
\[
\boxed{
a^{\frac32 (1 + w)} \propto t
}
\]
quindi nei vari casi abbiamo
\[
w = 0: \qquad a \propto t^{\frac23}
\]
\[
w = \frac13 : \qquad a \propto t^{\frac12}
\]
\marginpar{Nel caso della costante cosmologica l'integrale non è quello di un polinomio}
\[
w = -1: \qquad a \propto e^{\alpha t}
\]
\begin{oss}
	$w = -\frac13 $ è la linea di separazione tra l'universo che accelera e quello che decelera.
\end{oss}

\section{Universo di Einstein}
Consideriamo le equazioni di Einstein con il termine cosmologico
\[
R_{\mu \nu}  - \frac12 g_{\mu \nu} R + \Lambda g_{\mu \nu} = 8 \pi G T_{\mu \nu}
\]
Possiamo tuttavia condensare il termine cosmologico nel tensore energia impulso e riscrivere quest'ultimo come
\[
\tilde{T}_{\mu \nu} = T_{\mu \nu} - \frac{\Lambda}{8 \pi G} g_{\mu \nu}
\]
Quindi
\[
R_{\mu \nu} - \frac12 g_{\mu \nu} R = 8 \pi G \tilde{T}_{\mu \nu}
\]
Possiamo inoltre interpretare il termine cosmologico come un fluido con $w = -1$.\\
Einstein suppone un universo statico, cioè $\dot{a} = \ddot{a} = 0$. Andando a sostituire nelle equazioni di Friedmann troviamo
\[
\tilde{\rho} = - 3 \tilde{p}
\]
\[
\tilde{\rho} = \frac{3k}{8 \pi G a^2} 
\]
Per un universo dominato dalla materia($w = 0$)
\[
\tilde{p} = 0 - \frac{\Lambda}{8 \pi G} = -\frac{k}{8 \pi G a^2} 
\]
quindi 
\[
\Lambda = \frac{k}{a^2}
\]
Chiamiamo $a_E = \frac{1}{\sqrt{\Lambda}}$ raggio di Einstein, dove abbiamo posto $k = 1$.
\marginpar{$k=1$ dalla seconda equazione di Friedmann} \\
La soluzione data da Einstein è però instabile, infatti, perturbando di poco $a_E \to a_E + \delta a$ osserviamo che l'equazione di Friedmann diventa
\[
\delta \ddot{a} = -\frac{4 \pi G}{3} a (\delta \tilde{\rho} + 3 \delta \tilde{p})
\]
Ricordiamo adesso che $\tilde{p} = -\frac{\Lambda}{8 \pi G}$, $\tilde{\rho} = \rho_M + \frac{\Lambda}{8 \pi G}$, $\rho_M \propto \frac{1}{a^3}$. Quindi $\delta \tilde{p} = 0, \; \delta \tilde{\rho} = \delta \rho_M \propto -\delta a$. Allora
\[
\delta \ddot{a} \propto \delta a
\]
Quindi per una piccola perturbazione, a esplode.

\section{Universo di de Sitter}
Consideriamo un universo vuoto, in cui quindi gli unici contributi al tensore energia impulso sono dovuti alla costante cosmologica.
\[
\tilde{\rho} = -\tilde{p} = \frac{\Lambda}{8 \pi G}
\]
Supponendo $k=0$ dalla seconda equazione di Friedmann otteniamo
\[
\frac{\dot{a}}{a} = \sqrt{\frac{\Lambda}{3}}
\]
quindi un'accelerazione eterna.

\section{Densità critica e curvatura}
Ricordiamo la seconda equazione di Friedmann
\[
(\frac{\dot{a}}{a})^2 + \frac{k}{a^2} = \frac{8 \pi G}{3} \rho
\]
Se definiamo
\marginpar{al tempo attuale $\rho_{crit}$ vale $1.05\cdot 10^4 h^2 \frac{eV}{cm^2}$} 
\[
\rho_{crit} = \frac{3 H^2}{8 \pi G}
\]
Allora possiamo riscrivere l'equazione come
\[
\frac{k}{a H^2} = \frac{8 \pi G}{3 H^2}\rho - (\frac{\dot{a}}{a})^2 \frac{1}{H^2}
\]
\marginpar{$\Omega = \frac{\rho}{\rho_{crit}}$ \\ $\Omega_k = -\frac{k}{a H^2}$}
\[
\ra \Omega - 1 = -\Omega_k
\]
Possiamo allora misurare la densità di materia $\Omega$ per determinare la curvatura k. \\
Analogamente possiamo definire $\Omega_M = \frac{\rho_M}{\rho_{crit}},\, \Omega_R = \frac{\rho_R}{\rho_{crit}}, \, \Omega_\Lambda = \frac{\rho_\Lambda}{\rho_{crit}}$
\marginpar{$\Omega_M \propto a^{-3}$ \\ $\Omega_R \propto a^{-4}$ \\ $\Omega_\Lambda = c.te$}.\\
In generale avremo che $\rho = \sum_i \rho_i$, allora
\[
\rho = \frac{3 H^2}{8 \pi G} \{\Omega_\Lambda + \Omega_M + \Omega_R \}
\]
Sostituendo $\rho$ otteniamo la relazione di consistenza
\[
\Omega_M +\Omega_R +\Omega_\Lambda +\Omega_k = 1
\]
Definiamo poi il parametro di accelerazione
\marginpar{$q > 0$: decelerazione \\ $q<0$ : accelerazione}
\[
q = -\frac{\ddot{a}}{a H^2} = -\frac{\ddot{a} a}{\dot{a}^2}
\]
In questo modo possiamo riscrivere la prima equazione di Friedmann dividendo entrambi i membri per $H^2$
\[
q = \frac{4 \pi G}{3 H^2} \rho (1 + 3w)
\]
\marginpar{$w > -\frac13$: decelerazione \\ $w < -\frac13$: accelerazione}
\[
\ra q = \frac12 \Omega (1 + 3w)
\]
\begin{oss}[Problema della costante cosmologica]
	La densità di energia di punto zero è 
	\[
	<\rho>_{p0} = \int_{0}^{k_{max}} \frac12 k \frac{d^3 k}{(2 \pi )^3} = \frac{k_{max}^4}{16 \pi^2}
	\]
	dove $k_max$ è il cutoff ultravioletto della teoria che stiamo considerando.\\
	Nel caso della gravità quantistica $k_{max} \sim 10^{19} \, GeV \ra <\rho>_{p0} \sim 10^{73} \, GeV^4$\\
	Per la QCD $k_{max} \sim 100 \, MeV \ra <\rho>_{p0} \sim 10^{-6}\, GeV^4$\\
	Per il modello standard $<\rho>_{p0} \sim 10^{6}\, GeV^4$.
\end{oss}

Possiamo scrivere $\Omega_M$ (ed analogamente le altre $\Omega$) come
\[
\Omega_M = \frac{\rho_M}{\rho_{crit}} = \frac{1}{\rho_{crit}} \rho_{M, 0} \bigg(\frac{a_0}{a(t)} \bigg)^3 =  \frac{\rho_{crit, 0}}{\rho_{crit}} \frac{\rho_{M, 0}}{\rho_{crit, 0}} \bigg(\frac{a_0}{a(t)} \bigg)^3 = \frac{\rho_{crit, 0}}{\rho_{crit}} \Omega_{M, 0}  \bigg(\frac{a_0}{a(t)} \bigg)^3
\]
Allora possiamo riscrivere anche l'espressione per $\rho$
\[
\boxed{
\rho = \rho_{crit, 0} \bigg[\Omega_\Lambda +  \Omega_{M, 0}  \bigg(\frac{a_0}{a(t)} \bigg)^3 +  \Omega_{R, 0}  \bigg(\frac{a_0}{a(t)} \bigg)^4 \bigg]
}
\]
Riprendiamo poi la seconda equazione di Friedmann e dividiamo tutto per $a_0^2 H_0^2$. Otteniamo allora
\[
\bigg(\frac{\dot{a}}{a_0}\bigg)\frac{1}{H_0^2} + \frac{k}{a_0^2 H_0^2} = \bigg(\frac{a}{a_0}\bigg)^2 \bigg[\Omega_\Lambda +  \Omega_{M, 0}  \bigg(\frac{a_0}{a(t)} \bigg)^3 +  \Omega_{R, 0}  \bigg(\frac{a_0}{a(t)} \bigg)^4 \bigg]
\]
Poniamo $x = \frac{a(t)}{a_0} = \frac{1}{1 + z}$ e $k=0$ ed otteniamo
\[
\bigg(\frac{H}{H_0}\bigg)^2 = \bigg[\Omega_\Lambda +  \Omega_{M, 0}  x^{-3} +  \Omega_{R, 0}  x^{-4} \bigg]
\]
Chiamiamo
\[
\boxed{
	E^2(x) = \Omega_\Lambda +  \Omega_{M, 0}  x^{-3} +  \Omega_{R, 0}  x^{-4} 
}
\]
allora
\[
\frac{1}{a} \frac{da}{dt} = H_0 E(x)
\]
Cambiando variabile e ricordando che $dx = \frac{1}{a_0} da$ otteniamo che
\[
dt = \frac{dx}{H_0 E(x) x}
\]
Quindi il tempo dal Big Bang (tempo 0) ad un certo $x$ si calcola come
\[
\boxed{
	t(x) = \frac{1}{H_0}\int_{0}^{x} \frac{1}{x E(x)}\,dx
}
\]
L'età dell'universo sarà allora
\[
t_0= \frac{1}{H_0}\int_{0}^{1} \frac{1}{x E(x)}\,dx
\]
\begin{ex}
	Calcoliamo l'età dell'universo nel caso di quello di Einstein - de Sitter
	\[
	\Omega_{M, 0}= 1, \Omega_{R, 0} = 0, \Omega_\Lambda = 0
	\]
	\[
	t_0 = \frac{1}{H_0} \int_{0}^{1} \frac{dx}{x \cdot x^{-\frac32}} = \frac23 H_0^{-1} = 7\cdot 10^9 h^{-1} yrs = 9.5 \cdot 10^9 yrs
	\]
	Tuttavia confrontando questo valore con la vita media dei radionuclei o delle nane bianche si osserva che l'età è troppo bassa.\\
	La luminosità minima delle nane bianche è $3\cdot10^{-5} L_{solare}$, allora queste stelle non hanno avuto abbastanza tempo per raffreddarsi, secondo il modello di Einstein. Dalla luminosità di queste stelle si è infatti stimato che $t_0 = 10 \pm 2\, Gy$, mentre dalla CMB si stima $t_0 = 13.7 \pm 0.2\, Gy$
\end{ex}
\section{Modello $\Lambda$CDM}
Nel modello $\Lambda$CDM abbiamo che $\Omega_\Lambda + \Omega_{M, 0} = 1$ quindi
\marginpar{Inserire immagine}
\[
t_0 = \frac{1}{H_0} \int_0^1 \frac{dx}{x(\Omega_\Lambda + \Omega_{M, 0} x^{-3})^{\frac12}} = \frac23 \frac{1}{H_0 \Omega_\Lambda^\frac12 } \ln\bigg(\frac{1 + \Omega_\Lambda^\frac12}{\sqrt{1 - \Omega_\Lambda}} \bigg)
\]

\section{Orizzonte di Particella}
Consideriamo adesso dei fotoni ($ds^2=0$) allora dalla metrica di Friedmann otteniamo che 
\[
\frac{dt}{a(t)} = \frac{dr}{\sqrt{1 - kr^2}}
\]
Se integriamo nel tempo tra 0 e $t_0$ questo corrisponderà ad integrare nel raggio tra 0 ed $r_{max}$.\\
Nel caso di cosmologie standard, cioè dominate da radiazione e materia, $a(t) \propto t^n, \, n= \frac23 \frac{1}{1 + w}$. L'integrale temporale e di conseguenza quello spaziale convergono se $n<1$ cioè se $w > -\frac13$, quindi per un universo che decelera. \marginpar{$\rho + 3p > 0$ è detta Strong Energy Condition}
\\
La distanza fisica d è poi
\[
d(t) = a(t) \int_{0}^{r(t)} \frac{dr'}{\sqrt{1 - kr'^2}} = \int_{0}^{t}dt' = \int_{0}^{x(t)} \frac{dx}{x E(x) H_0} 
\]
Se integriamo nel tempo tra 0 ed 1 troviamo quindi la distanza fisica massima che definisce la massima distanza a cui si può trovare una particella ed è detta orizzonte di particella.

\section{Orizzonte degli eventi}
Esistono tuttavia delle zone dell'universo da cui non possiamo ricevere informazioni, dobbiamo perciò riconsiderare gli estremi di integrazione
\[
\int_{t_*}^{\infty} \frac{dt}{a(t)} = \int_{0}^{r_*} \frac{dr}{\sqrt{1 - kr^2}}
\]
Se $r > r_*$ allora l'evento nella posizione r non sarà mai osservabile.\\
$r_*$ è detto orizzonte degli eventi.

\begin{ex}
	Consideriamo l'universo di de Sitter, quindi $a(t) = a_0 e^{Ht}, \, k=0$ allora
	\[
	\frac{1}{a_0} \int_{t_*}^{\infty} \frac{dt}{e^{Ht}} = \int_{0}^{r_*}dr = r_*
	\]
	Allora
	\[
	r_* = \frac{1}{a_0 H}e^{-Ht_*}
	\]
\end{ex}
Se $a \propto t^n$, l'integrale non converge per $w > -\frac13$ quindi non c'è orizzonte degli eventi.

\section{Distanza di Luminosità}
Ricordiamo che 
\[
l = \frac{L}{4 \pi d^2}
\]
nello spazio euclideo. \\ 
Tuttavia in presenza dell'espansione dell'universo l'espressione cambia.
\\
Per effetto geometrico 
\[
\frac{1}{d^2} \to \frac{1}{a^2(t_0) r_s^2}
\]
con $t_0$ tempo attuale ed $r_s$ raggio della sorgente. \\
C'è poi un fattore $\frac{1}{1 + z}$ dovuto al redshift. \\
Infine dobbiamo tener conto anche della dilatazione temporale: l'intervallo di tempo tra l'emissione di due fotoni è
\[
\frac{\Delta t_0}{\Delta t_s} = \frac{a(t_s)}{a(t_0)} = \frac{1}{1 + z}
\]
Combinando questi 3 effetti otteniamo la distanza di luminosità $d_L = a(t_0) r_s (1 + z)$ t.c. 
\[
l = \frac{L}{4 \pi d_L^2}
\] 

\section{Legge di Hubble al second'ordine}
\[
z = -1 + \frac{a(t_0)}{a(t_s)} \simeq -1 + \frac{1}{1 + H_0(t_s - t_0) - \frac12 H_0^2 q_0 (t_s - t_0)^2} 
\]
\[
\ra z \simeq -1 + 
 \bigg[
 	 \bigg(
 	 1 + H_0 (t_0 - t_s) + \frac12 H_0^2 q_0 (t_0 - t_s)^2 
 	 \bigg)
	  + H_0^2 (t_0 - t_s)^2 
\bigg] = H_0 (t_0 - t_s) + \frac12 H_0^2 (q_0 + 2) (t_0 - t_s)^2
\]
Invertendo l'equazione otteniamo che
\[
 H_0 (t_0 - t_s) = z - \frac12 (q_0 + 2)z^2
\]
Ricordiamo adesso che 
\[
f_k \equiv \int_{0}^{r_s} \frac{dr}{\sqrt{1 - kr^2}} \underbrace{\simeq r_s}_{\text{al prim'ordine }}
\]
e che
\[
\int_{t_s}^{t_0} \frac{dt}{a(t)} = \int_{0}^{r_s} \frac{dr}{\sqrt{1 - kr^2}} \simeq r_s
\]
Posso scrivere il LHS come
\marginpar{Sviluppando al prim'ordine $\frac{a_0}{a(t)}$}
\[
\int_{t_s}^{t_0} \frac{dt}{a(t)} = \frac{1}{a_0} \int_{t_s}^{t_0} \frac{a_0}{a(t)}\,dt = \frac{1}{a_0} \int_{t_s}^{t_0}  (1 + H_0 (t_0 - t)) = \frac{1}{a_0} \bigg[(t_0 - t_s) + \frac{H_0}{2} (t_0 - t_s)^2 \bigg]
\]
Quindi
\[
r_s = \frac{1}{a_0} \int_{t_s}^{t_0}  (1 + H_0 (t_0 - t)) = \frac{1}{a_0} \bigg[(t_0 - t_s) + \frac{H_0}{2} (t_0 - t_s)^2 \bigg]
\]
In precedenza avevamo trovato che
\[
t_0 - t_s = \frac{1}{H_0} (z - (1+ \frac{q_0}{2}) z^2)
\]
Allora
\[
d_L^2 = a^2(t_0) r_s^2 (1 + z)^2 = a_0^2  \frac{1}{a_0^2 H_0^2} \bigg( z - \frac{z^2}{2} (1 + q_0)\bigg)^2(1 + z)^2
\]
\[
\ra H_0 d_L = \bigg( z - \frac{z^2}{2} (1 + q_0)\bigg)(1 + z)
\]
che è la legge di Hubble al secondo ordine.

\section{Distanza di diametro angolare}
Vogliamo descrivere una certa area dello spazio in coordinate sferiche. Nel caso euclideo avremmo, come è noto, $ds^2 = d^2 d\Omega^2$, tuttavia nella metrica FRW questa espressione diventerà $ds^2 = d_A^2 d\Omega^2$, con $ds^2(r_s) = f_k(r_s)^2 a^2(t_s) d\Omega^2$. Possiamo allora ricavare $d_A^2$ ad una distanza pari ad $r_s$ invertendo l'equazione. Otteniamo così
\[
d_A^2 = \frac{ds^2}{d\Omega^2} = f_k^2(r_s) a^2(t_s)
\]
che per $k=0$ si riduce a 
\[
d_A^2 = r_s^2 a^2(t_s) = \frac{r_s^2 a^2(t_0)}{(1 + z)^2}
\]
Allora
\[
\frac{d_A}{d_L} = \frac{1}{(1+ z)^2}
\]
\[
H_0 d_A \simeq z - \frac12 (3 + q_0)z^2
\]
dove abbiamo utilizzato lo sviluppo al prim'ordine di $\frac{1}{1 + z}$
\\
Definendo 
\marginpar{S è l'area, $\mathcal{L} = \frac{L}{S}$ è la luminosità assoluta per unità di area}
\[
B = \frac{l}{\Omega} = \frac{L}{4 \pi d_L^2} \cdot \frac{d_A^2}{S}
\]
superficie di brillanza, otteniamo che
\[
B = \frac{\mathcal{L}}{4 \pi} \bigg(\frac{d_A}{d_L}\bigg)^2 \propto \frac{1}{(1 + z)^4}
\]
Se, tuttavia, il redshift fosse dovuto ad un qualche tipo di polvere, avremmo
\[
d_L = d(1 + z)
\]
poichè non ci sarebbe alcun effetto di dilatazione temporale. Quindi
\[
\frac{d_A}{d_L}  \propto\frac{1}{1 + z}
\]
cioè
\[
B \propto \frac{1}{(1 + z)^2}
\]
Calcoliamo adesso $d_L$ in funzione della composizione dell'universo. Sappiamo che 
\[
dt = \frac{1}{H_0} \frac{dx}{x E(x)}
\]
e per la luce
\[
\int_{t(z)}^{t_0} \frac{dt'}{a(t')} = \int_{0}^{r(z)} \frac{dr'}{\sqrt{1 - kr'^2}}
\]
\marginpar{Ricordiamo che $x = \frac{a(t)}{a_0}$}
\[
\ra r(z) = S\bigg[ \int_{t(z)}^{t_0} \frac{dt'}{a(t')} \bigg] = S\bigg[\frac{1}{a_0 H_0} \int_{\frac{1}{1 + z}}^{1} \frac{dx}{x^2 E(x)}\bigg]
\]
con 
\[
S[y] = \begin{cases}
	\sin(y),& k=1 \\
	y,& k=0 \\
	\sinh(y),& k=-1 \\
\end{cases}
\]
Usando il fatto che $\Omega_k = - \frac{k}{a_0^2 H_0^2}$ si ha che
\marginpar{Ricordiamo che $\sinh(ix) = i \sin(x)$}
\[
a_0 r(z) = \frac{1}{H_0 \Omega_k} \sinh\bigg(\sqrt{\Omega_k}\int_{\frac{1}{1 + z}}^{1} \frac{dx}{x^2 E(x)}  \bigg)
\]
da cui possiamo ricavare $d_L(z)$ usando la relazione
\marginpar{Inserire immagine}
\[
d_L(z) = a_0 r(z) (1 + z)
\] 
\section{Ricerca di candele standard}
La relazione di Tully-Fisher ci dice che: \\
Per galassie a spirale
\[
L \propto v_{rot}^4
\]
Per galassie ellittiche
\[
L \propto \Delta v_{rot}^4
\]
Alternativamente possiamo usare le supernove di tipo 1a per determinare le candele standard. Le supernove di tipo 1a sono infatti un sistema binario costituito da una nana bianca ed un'altra stella che viene assorbita dalla NB. Quando la massa della NB supera la massa di Chandrasekhar ($1.4$ masse solari), il sistema collassa producendo radiazioni termo-nucleari esplosive e dando origine ad una supernova.\\
\marginpar{In forma luminosa l'energia delle supernove è di circa $10^{51}$ erg}
Studiando le curve di luce $l(t$) delle supernove possiamo determinare $L(t)$.

\section{Magnitudine}
\begin{defn}[Magnitudine]
	Definiamo la magnitudine apparente o relativa come
	\[
	m = -2.5 \log\bigg(\frac{l}{L_\odot}\bigg) + c
	\]
	La magnitudine assoluta è poi la magnitudine relativa calcolata alla distanza di 10 pc, cioè
	\[
	M = -2.5 \log(L) + c
	\]
	\[
	\mu := M - m = 5 \log\bigg(\frac{d_L}{10 \, pc} \bigg)
	\]
\end{defn}

Se chiamiamo adesso $z_\Lambda$ il momento in cui $\Omega_\Lambda$ diventa dominante rispetto a $\Omega_M$ vediamo che $z_\Lambda$ risolve 
\[
\Omega_\Lambda = \Omega_{M, 0} (1 + z_\Lambda)^3
\]
\marginpar{$\Omega_\Lambda = 0.25$ \\ $\Omega_{M, 0} = 0.75$}
\[
\ra z_\Lambda = 0.44
\]
\marginpar{?}
Questo viene detto problema della coincidenza cosmica.

\section{Natura energia oscura}
L'energia oscura è un fluido a densità negativa di cui non si conosce la natura. Vediamo alcune possibili soluzioni a questo problema.\\
\begin{enumerate}
	\item Possiamo introdurre un nuovo tipo di materia come ad esempio un campo scalare con $w < -\frac13$. Se la densità di energia dipende dal tempo, cioè questo termine ha una dinamica, viene chiamato \textit{quintessenza}.
	\item Possiamo modificare la relatività generale, in particolar modo l'azione
	\marginpar{Se la correzione esiste deve essere molto piccola in quanto la RG è molto precisa}
	\[
	S = \frac{1}{16 \pi G} \int d^4 x\, \sqrt{-g}F(\varphi) R
	\]
	con $\varphi$ campo scalare. Alternativamente possiamo introdurre uno schermaggio della gravità che si ottiene introducendo un termine di massa per i gravitoni.
\end{enumerate}
La quintessenza è un campo scalare Q con azione
\[
S = \frac12 \partial_\mu Q \partial^\mu Q + V(Q)
\]
Il suo tensore energia impulso sarà dunque
\[
T^{\mu \nu} = \partial^\mu Q \partial^\nu Q - g^{\mu \nu}\bigg(\frac12 \partial_\alpha Q \partial^\alpha Q + V(Q)\bigg)
\]
Per cui confrontandolo con la nota espressione
\[
T_{\mu \nu} = (\rho + p)u_\mu u_\nu + p g_{\mu \nu}
\]
troviamo che
\[
p = -\frac12 g^{\mu \nu} \partial_\mu Q \partial_\nu Q - V(Q)
\]
\[
(\rho + p) u^\mu u^\nu = \partial^\mu Q \partial^\nu Q
\]
Moltiplicando a destra e a sinistra per $g_{\mu \nu}$ ed usando che $u^\mu u_\mu = -1$ troviamo che
\[
\rho = -\frac12 g^{\mu \nu} \partial_\mu Q \partial_\nu Q + V(Q)
\]
Allora, supponendo che il termine potenziale domini su quello cinetico (condizione di slow roll), si ha che
\[
w = -\frac{p}{\rho} \to -1
\]
$\rho$ e p devono poi soddisfare anche l'equazione di conservazione
\[
\dot{\rho} +3H(\rho + p)=0
\]
che diventa
\marginpar{Vedi Weinberg}
\[
\ddot{Q} + 3 H \dot{Q} + \dfrac{\partial V}{\partial Q} = 0
\]
Soluzioni di questa equazione sono del tipo 
\[
V(Q) = M^{4 + \alpha} Q^\alpha \qquad \alpha > 0
\]
Tuttavia non c'è una ragione specifica per credere che il potenziale possa avere questa forma.

\section{Problema materia oscura}
Il modello $\Lambda$CDM richiede che $\Omega_{M, 0} = 0.3$, tuttavia si osserva che la materia barionica (galassie, elettroni, ...) non è così tanta ($\Omega_B \sim 0.05$) \\
Possiamo studaire l'universo come un sistema statistico, cioè un gas di stelle che, sottoposto alla mutua gravità, si comprime e si riscalda. \\
Studiando la velocità media delle particelle in campo gravitazionale mediante il teorema del viriale si può capire se la massa è corretta oppure c'è della massa \textit{nascosta}.
\[
2\cdot M <v^2> +  G M^2 <\frac1d> = 0
\]
Allora la massa dinamica M, cioè tutta la massa che genera il campo gravitazionale è
\[
M = \frac{2 <v^2>}{G <\frac1d>}
\]
dove $<v^2>$ si stima mediante effetto Doppler. \\
In generale per un cluster di galassie $v \sim 10^3 \frac{km}{s}$ quindi 
\[
\bigg(\frac{M}{L}\bigg)_{\text{cluster}} \sim 500\bigg(\frac{M}{L}\bigg)_\odot
\]
Tuttavia per il coma cluster si è misurato 
\[
\bigg(\frac{M}{L}\bigg)_{\text{c.c.}} \sim 10\bigg(\frac{M}{L}\bigg)_\odot
\]
che a parità di luminosità (supponiamo che la materia oscura non emetta) ci dice che c'è della massa nascosta, quindi ci deve essere qualcos'altro che contribuisca al potenziale gravitazionale.
\begin{oss}[Stima di $\frac{M}{L}$ a piccoli redshift]
	Studiamo la dipendenza di $\frac{M}{L}$ da $H_0$.\\
	\[
	M \propto <\frac1d>^{-1} \propto \frac{z}{H_0}
	\]
	\marginpar{Dalla legge di Hubble $H_0 d = z$}
	\[
	L \propto d^2 \propto H_0^{-2}
	\]
	Allora
	\[
	\frac{M}{L} \propto H_0
	\]
\end{oss}

Stimiamo adesso $\Omega_{M, 0}$.
\[
\rho_{M, 0} = \frac{M}{V} = \frac{M}{L} \mathcal{L}
\]
dove $\mathcal{L} \sim 2\cdot 10^8 h L_\odot\, Mpc^{-3}$. Allora
\[
\Omega_{M, 0} = \frac{\rho_{M, 0}}{\rho_{crit, 0}} = \frac{M}{L} \frac{\mathcal{L}}{\rho_{crit, 0}} \sim 0.15
\]
I gas ionizzati contenuti nei cluster possono spiegare il valore di $\Omega_{M, 0}$?\\
Usando il fatto che 
\[
\frac12 M <v^2> = k_B T
\]
con $<v> \sim 10^3 \frac{km}{s}$ si ottengono temperature nel range di $10^7 - 10^8 \, K$ che corrispondono all'emissione di raggi X.
\section{Curve di rotazione delle galassie}
Ci aspettiamo che all'equilibrio gravitazionale 
\[
\frac{m v^2}{r} = \frac{G M(<r)m}{r^2}
\]
con 
\[
M(<r) = \int_{0}^{r} 4 \pi \rho(r) r^2 \, dr
\]
Allora 
\[
v = \sqrt{\frac{G M(<r)}{r}}
\]
Ci aspettiamo che per $r > r_*$, M diventi costante e quindi $v \propto \sqrt{\frac1r}$, tuttavia la curva di rotazione delle galassie si appiattisce. Questo ci suggerisce che ci debba essere qualcos'altro, che chiamiamo alone di materia oscura.
\section{MACHOS (Massive Astrophysical Compact Halo ObjectS)}
Sono i candidati alla materia oscura, ma barionica. Sono stati ricercati con il microlensing, ma la ricerca è fallita.
\marginpar{inserire immagine}

Le osservazioni non hanno trovato così tanti MACHOS e quindi si esclude che questi possano formare l'alone di materia oscura.

\section{MOND (MOdified Newtonian Dynamics)}
Modifichiamo la legge di Newton
\[
F = ma \to F = 
\begin{cases}
	ma, & a>a_0 \\
	\frac{ma^2}{a_0}, & a< a_0
\end{cases}
\]
con $a_0 \sim 10^{-10} \frac{m}{s^2}$. \\
Sorprendentemente questo modello spiega in maniera accurata le curve di rotazione.

\section{Termodinamica del primo universo}
Ricordiamo che le funzioni di distribuzione per bosoni e fermioni sono, ponendo $k_B=1$
\[
F_A (E, T) = \frac{g_A}{(2 \pi)^3} \frac{1}{exp[(E - \mu)/T] \pm 1} =  \frac{g_A}{(2 \pi)^3} f_A(E, T)
\]
Allora la densità di particelle di una specie i sarà 
\[
\boxed{
n_i = \int F_i(p, T) d^3p =  \frac{g_i}{(2 \pi)^3} \int f_i(p, T) d^3 p =  \frac{g_i}{2 \pi^2} \int_{m}^{\infty}   \frac{E \sqrt{E^2 - m^2}}{exp[(E - \mu)/T] \pm 1}\, dE
}
\]
la densità di energia
\[
\boxed{\rho_i = \int F_i(p, T) E d^3p = \frac{g_i}{2 \pi^2} \int_{m}^{\infty}   \frac{E^2 \sqrt{E^2 - m^2}}{exp[(E - \mu)/T] \pm 1}\, dE}
\]
e la pressione
\[
\boxed{p_i = \int F_i(p, T) \frac{p^2}{3E} d^3p = \frac{g_i}{6 \pi^2} \int_{m}^{\infty}   \frac{ (E^2 - m^2)^\frac32}{exp[(E - \mu)/T] \pm 1}\, dE}
\]
Definiamo adesso il limite ultrarelativistico quando $T \gg m , \mu$.
\\
In questo limite la densità di energia per i bosoni è 
\[
\rho_{bosoni} = \frac{g}{2 \pi^2} \int_{0}^{\infty}   \frac{E^3}{exp[E/T] - 1}\, dE
\]
Chiamando $x = E/T$ otteniamo
\marginpar{$$\int_{0}^{\infty} \frac{x^{n-1}}{e^x - 1} = \zeta(n) \Gamma(n)$$}
\marginpar{$$\zeta(2) = \frac{\pi^2}{6}$$ \\ $$\zeta(4) = \frac{\pi^4}{90}$$ \\ $$\zeta(6) = \frac{\pi^6}{945}$$}
\[
\rho_{bosoni} = \frac{g T^4}{2 \pi^2} \int_{0}^{\infty}   \frac{x^3}{e^x - 1}\, dE =  \frac{g T^4}{2 \pi^2} \zeta(4) \Gamma(4) = \frac{\pi^2}{30}g T^4
\]
Analogamente la densità di bosoni ultrarelativistica è
\[
n_{bosoni} = \frac{g}{2 \pi^2} T^3 \Gamma(3) \zeta(3)
\]
mentre per la pressione basta ricordarci che per un fluido ultrarelativistico $p = \frac13 \rho$.\\
Nel caso dei fermioni, invece, dobbiamo usare l'integrale notevole
\[
\int_{0}^{\infty} \frac{x^{n-1}}{e^x + 1} = B(n) \Gamma(n)
\]
\marginpar{$$B(2) = \frac{\pi^2}{12}$$ \\ $$B(4) = \frac78 \zeta(4)$$}
dove $\{B(n)\}$ sono i numeri di Bernoulli.\\
Allora nel caso ultrarelativistico
\[
\rho_F = \frac78 \rho_B 
\]
\[
p_F = \frac78 p_B
\]
Nel caso non relativistico ($T \gg m$), invece, dobbiamo usare l'approssimazione 
\[
E \simeq m + \frac{p^2}{2m}
\]
quindi
\[
n_i = \frac{g_i}{(2 \pi)^3} \int \frac{1}{exp[(E - \mu)/T]\pm 1} \, d^3p \simeq \frac{g_i}{(2 \pi)^3} \int \frac{1}{exp[(\frac{p^2}{2m} + m - \mu)/T]\pm 1} \, d^3p  
\]
In questo caso possiamo trascurare il $\pm 1$ al denominatore riottenendo così la statistica di Boltzmann 
\marginpar{$$x = \frac{p^2}{2mT}$$ \\ $$\int_{0}^{\infty} x^n e^{-x}\, dx = \Gamma(n- 1)$$ }
\[
n_i = \frac{g_i}{2 \pi^2} e^{(\mu - m)/T} \int_{0}^{\infty} p^2 e^{-\frac{p^2}{2mT}} \, dp = \frac{g_i}{2 \pi^2} e^{(\mu - m)/T} (mT)^{\frac32} \Gamma\bigg(\frac32\bigg) 
\]
quindi
\[
\boxed{
n_i = g_i \bigg(\frac{mT}{2 \pi} \bigg)^{\frac32}  e^{(\mu - m)/T}
}\]
Inoltre 
\[
p = nT \ll \rho
\]
Calcoliamo adesso le energie medie nel caso ultrarelativistico
\[
<E> = \frac{\rho}{n} = \begin{cases}
	\frac{\pi^4}{30 \zeta(3)}T \simeq 2.7 T,& \text{bosoni} \\
	\vspace{1mm} \\
	\frac{7 \pi^4}{180 \zeta(3)}T \simeq 3.15 T,& \text{fermioni} \\
\end{cases}
\]
Nel caso non relativistico abbiamo invece che
\[
<E> = m + \frac32 T
\]
\begin{ex}
	Consideriamo il seguente processo
	\[
	e^+ + e^- \longleftrightarrow \gamma + \gamma
	\]
	Poichè il potenziale chimico all'equilibrio si conserva ed il potenziale chimico dei fotoni è zero (perchè il numero di fotoni non si conserva), abbiamo che 
	\[
	\mu_+ + \mu_- = 0
	\]
	Allora
	\[
	n_+ - n_- = \frac{g}{2 \pi^2} \int_0^\infty dE\, E\sqrt{E^2 - m^2} \bigg[\frac{1}{exp[(E - \mu)/T] - 1} - \frac{1}{exp[(E + \mu)/T] - 1}  \bigg]
	\]
	Quindi
	\[
	n_+ - n_- = \begin{cases}
		\frac{g T^3}{6 \pi^2} \bigg[\pi^2 \big(\frac{\mu}{T}\big) + \big(\frac{\mu}{T}\big)^3 \bigg], & T \gg m \\
		\vspace{0.5mm} \\
		2g\bigg(\frac{mT}{2 \pi} \bigg)^\frac32 e^{-m/T} \sinh\big(\frac{\mu}{T}\big), & T \ll m \\
	\end{cases}
	\]
\end{ex}
Ricordiamo ora l'espressione per la densità di energia ultrarelativistica dei bosoni (i fermioni hanno $\frac78$ davanti).
\[
\rho_{bosoni} = \frac{\pi^2}{30}g T^4
\]
Possiamo allora unirle per calcolare la densità di energia per un qualunque gas ultrarelativistico composto da più specie. Otteniamo così
\[
\rho = \frac{\pi^2}{30}g_*(T) T^4
\]
dove
\marginpar{T è la temperatura del bagno termico, mentre $T_i$ è la temperatura della specie i-esima}
\[
g_*(T) = \sum_{i, bosoni} g_i \bigg(\frac{T_i}{T}\bigg)^4 + \frac78 \sum_{i, fermioni} g_i \bigg(\frac{T_i}{T}\bigg)^4
\]
è il numero efficace di gradi di libertà.
\begin{oss}
	Stiamo trascurando il contributo delle specie non relativistiche
\end{oss}

Usando poi l'equazione di Friedmann con $k=0$ troviamo che
\marginpar{Q. perchè $k=0$?}
\[
\bigg(\frac{\dot{a}}{a}\bigg)^2 = H^2(t) = \frac{8 \pi G}{3}\rho_{tot} = \frac{8 \pi G}{3} \frac{\pi^2}{30}g_*(T) T^4
\]
Allora
\marginpar{$m_{Pl} =  1.22 \cdot 10^{19} GeV$}
\[
\boxed{
	H(t) = 1.66 \sqrt{g_*(T)} \frac{T^2}{m_{Pl}}
}
\]
\begin{ex}
	Nel caso di universo dominato da radiazione l'età dell'universo è
	\[
	t_0 = \frac{1}{H_0} \int_{0}^1 dx \, \frac{1}{x \cdot x^{-2}} = \frac{1}{2 H_0}
	\]
\end{ex}
\begin{ex}
	\marginpar{100 Mev è l'energia della transizione di fase quark-adronica}
	Calcoliamo $g_*(T)$ per $T \in [1\, MeV, 100\, Mev]$. A questo range di temperature contribuiscono $\gamma, \nu, e^+, e^-$ quindi all'equilibrio termico
	\[
	g_*(T) = g_\gamma + \frac78 (g_{e+} + g_{e-} + g_\nu \cdot 6)
	\]
	dove 6 è il numero di specie di neutrini che stiamo considerando (3 neutrini, 3 antineutrini).
	\\
	\marginpar{La degenerazione per particelle di spin s è $2s + 1$, mentre per i neutrini la degenerazione è 1 perchè hanno elicità fissata.}
	$g_\gamma = 2, \quad g_{e+} = g_{e-} = 2, \quad g_\nu = 1$ \\
	Quindi
	\[
	g_*(T) = 10.75
	\]
\end{ex}

\section{Densità di entropia}
L'entropia S è costante all'equilibrio, possiamo allora definire una densità di entropia s tale che
\[
S = s V = s a^3 = c.te \ra s \propto \frac{1}{a^3}
\]
Sappiamo poi che
\[
dU = TdS - pdV \ra Td(sV)  = d(\rho V) + pdV \ra TVds + TsdV = Vd\rho +  \rho dV + pdV
\]	
Allora
\[
s(T) = \frac{\rho + p}{T}
\]
\[
p = \frac13 \rho \ra s(T) = \frac43 \frac{\rho(T)}{T} = \frac43 \frac{\pi^2}{30} g_{*s}(T) T^3 
\]
dove
\[
g_{*s} =  \sum_{i, bosoni} g_i \bigg(\frac{T_i}{T}\bigg)^3 + \frac78 \sum_{i, fermioni} g_i \bigg(\frac{T_i}{T}\bigg)^3
\]
Se l'entropia è conservata allora
\[
g_{*s} T^3 a^3 = c.te \ra T \propto \frac{1}{a} g_{*s}^{-\frac13}(T)
\]

\section{Disaccoppiamento delle specie}
Se consideriamo un certo processo allora il suo rate caratteristico sarà
\[
\Gamma = n \sigma v
\]
dove $\sigma$ è la sezione d'urto.\\
Per capire se una specie è in equilibrio o meno possiamo confrontare questo rate con il rate di espansione dell'universo. La temperatura a cui i due rate sono equivalenti è detta temperatura di disaccoppiamento $T_D$
\[
\Gamma(T_D) = H(T_D)
\]
Infatti dopo questa temperatura, la velocità delle particelle sarà minore di quella dell'espansione dell'universo e quindi la probabilità che i processi accadano diminuirà. Quando $T < T_D$ le interazioni non saranno più efficaci quindi la particella esce dall'equilibrio con il bagno termico. \\
Per $T \le T_D$ abbiamo che
\[
f(p, T) = \frac{1}{exp[(E - \mu)/T] \pm 1}
\]
ma per definizione
\[
f(p, T) = \frac{dn}{d^3p}
\]
con 
\[
n \propto a^{-3} \quad p^3 \propto a^{-3}
\]
Allora $f(p)$ è costante a T fissata.
\begin{oss}
	Il fatto che f non vari segue anche dal teorema di Liouville
\end{oss}
Per $T > T_D$, invece, non posso fissare la temperatura e studiare il comportamento della funzione al variare del tempo in quanto temperatura e tempo sono legati tra loro.
\\
Consideriamo adesso il caso ultrarelativistico con $T \le T_D$. Allora le particelle non interagiscono e la funzione di distribuzione non dipende dal tempo.
\\
L'energia subisce però il redshift ($p \propto a^{-1}$). Allora possiamo considerare il redshift anzichè sull'energia, sulla temperatura, per cui avremo che \[
f(p, T_D) = f(p_D, T_D) = f(p(t) \frac{a(t)}{a_D}, T_D) = f(p(t), T)
\]
con $T = T_D \frac{a_D}{a(t)}$. Infatti
\[
f(p(t) \frac{a(t)}{a_D}, T_D) \sim \bigg[exp\bigg(\frac{p(t) a(t)}{T_Da_D}\bigg)\bigg]^{-1} = \bigg[exp\bigg(\frac{p(t)}{T_Da_D / a(t)}\bigg)\bigg]^{-1} \sim f(p(t), T)
\]

\end{document}